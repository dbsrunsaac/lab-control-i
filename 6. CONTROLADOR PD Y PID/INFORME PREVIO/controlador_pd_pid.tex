\documentclass[conference]{IEEEtran}
\IEEEoverridecommandlockouts
% The preceding line is only needed to identify funding in the first footnote. If that is unneeded, please comment it out.
\usepackage{cite}
\usepackage{amsmath,amssymb,amsfonts}
\usepackage{algorithmic}
\usepackage{graphicx}
\usepackage{textcomp}
\usepackage{xcolor}
\usepackage{tabularx}
\usepackage{multirow}
\usepackage{graphics} % for pdf, bitmapped graphics files
\usepackage{subfig}
\usepackage{subcaption}
\usepackage{hyperref}
\usepackage{academicons}
\usepackage{xcolor}
\usepackage{listings}
\def\BibTeX{{\rm B\kern-.05em{\sc i\kern-.025em b}\kern-.08em
		T\kern-.1667em\lower.7ex\hbox{E}\kern-.125emX}}
% Gráficas en MATLAB
\usepackage{tikz, pgfplots}
% Color Enlace
\definecolor{colorEnlace}{RGB}{0, 0, 0}
\hypersetup{
	colorlinks=true,
	linkcolor=colorEnlace,
	citecolor=colorEnlace,
	urlcolor=colorEnlace,
	pdfauthor={Ruth Juana Espino Puma},
	pdftitle={}
}
\lstset{
	language=Matlab, % Define el lenguaje
	basicstyle=\ttfamily\small, % Tamaño de letra pequeño
	keywordstyle=\color{blue}, % Color de las palabras clave
	commentstyle=\color{green}, % Color de los comentarios
	stringstyle=\color{red}, % Color de las cadenas de texto
	numbers=left, % Muestra los números de línea a la izquierda
	numberstyle=\tiny\color{gray}, % Estilo de los números de línea
	tabsize=1,
	stepnumber=1, % Muestra un número en cada línea
	breaklines=true, % Ajuste automático de línea
	frame=single, % Borde alrededor del código
	xleftmargin=0em, % Elimina el margen izquierdo
	framexleftmargin=0em % Elimina el espacio dentro del marco izquierdo
}
% Control 
\usepackage{amsmath}
\begin{document}
	
	\title{Experiencia N°6 - Controlador PD y PID}
	% Ing. Diego Darcy Arredondo Huarac
	\author{	
		\IEEEauthorblockN{Ruth Juana Espino Puma}
		\IEEEauthorblockA{Universidad Nacional de San Antonio Abad del Cusco}
		\textit{Escuela Profesional de Ingeniería Electrónica}\\
		\textit{Laboratorio de Control I}\\
		184657 \\\\
		Cusco, Perú
	}
	
	\maketitle
	
	\begin{abstract}
		
	\end{abstract}
	
	\begin{IEEEkeywords}
		
	\end{IEEEkeywords}
	
	\section{Introducción}
	
	\section{Objetivos}
	
	\begin{itemize}
		\item Diseñar Controlador PD y PID usando el Lugar Geométrico de las Raíces.
		\item Implementar Controlador PD y PID usando el Lugar Geométrico de las Raíces.
	\end{itemize}
	
	\section{Usando el Lugar Geométrico de las Raíces diseñar Controlador PD y PID}
	
	\begin{itemize}
		\item Reducir el sobrepico a la mitad en función a los polos asignados mediante el diseño de un controlador PD y PID
	\end{itemize}
	
	\section{Simulación del sistema de control en lazo cerrado diseñado mediante MATLAB}	
	
	\subsection{Función Escalón}
	\subsection{Función Impulso unitario}
	
	\section{Mostrar las gráficas del ítem anterior, ¿Cuál es la explicación de estas graficas?}
	
	
	\bibliographystyle{IEEEtran}
	\bibliography{biblio}
\end{document}



































