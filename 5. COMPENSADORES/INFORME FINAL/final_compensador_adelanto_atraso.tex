\documentclass[conference]{IEEEtran}
\IEEEoverridecommandlockouts
\usepackage{cite}
\usepackage{amsmath,amssymb,amsfonts}
\usepackage{algorithmic}
\usepackage{graphicx}
\usepackage{textcomp}
\usepackage{xcolor}
\usepackage{tabularx}
\usepackage{multirow}
\usepackage{graphics} % for pdf, bitmapped graphics files
\usepackage{subfig}
\usepackage{subcaption}
\usepackage{hyperref}
\usepackage{academicons}
\usepackage{xcolor}
\usepackage{listings}
\def\BibTeX{{\rm B\kern-.05em{\sc i\kern-.025em b}\kern-.08em
		T\kern-.1667em\lower.7ex\hbox{E}\kern-.125emX}}
% Gráficas en MATLAB
\usepackage{tikz, pgfplots}
% Color Enlace
\definecolor{colorEnlace}{RGB}{0, 0, 0}
\hypersetup{
	colorlinks=true,
	linkcolor=colorEnlace,
	citecolor=colorEnlace,
	urlcolor=colorEnlace,
	pdfauthor={Ruth Juana Espino Puma},
	pdftitle={Controlador }
}
% Control 
\usepackage{amsmath}
\begin{document}
	
	\title{Experiencia N°5 - Diseño de Compensadores}
	
	\author{
		\IEEEauthorblockN{Ruth Juana Espino Puma}
		\IEEEauthorblockA{
			Estudiante de Ingeniería Electrónica \\
			Cusco, Perú \\
			184657@unsaac.edu.pe}
		\and
		\IEEEauthorblockN{Davis Bremdow Salazar Roa}
		\IEEEauthorblockA{
			Estudiante de Ingeniería Electrónica \\
			Cusco, Perú \\
			200353@unsaac.edu.pe
		}
		\and
		\IEEEauthorblockN{Ing. Darcy Arredondo Huarac}
		\IEEEauthorblockA{
			Laboratorio de Control I\\
			Cusco, Perú\\
			diego.arredondo@unsaac.edu.pe}
	}
	\maketitle
	
	\begin{abstract}
		
	\end{abstract}
	
	\begin{IEEEkeywords}
		
	\end{IEEEkeywords}
	
	 \section{Muestre el procedimiento del cálculo de las resistencias y capacitores del diseño del circuito de la Figura 1 para el caso Subamortiguado y Sobreamortiguado en lazo cerrado}
	 \section{Muestre las gráficas obtenidas en el simulador para el caso Subamortiguado y Sobreamortiguado en lazo cerrado}
	 \section{Muestre las gráficas obtenidas con el Osciloscopio implementado para el caso Subamortiguado y Sobreamortiguado en lazo cerrado.}
	 \section{Analice el error en estado estacionario en el caso Subamortiguado y Sobreamortiguado en lazo cerrado.}
	 \section{¿La respuesta hallada en forma teórica es igual o similar al circuito implementado?}
	\bibliographystyle{IEEEtran}
	\bibliography{biblio}
\end{document}